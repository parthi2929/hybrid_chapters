\documentclass[float=false, crop=false,11pt,oneside]{book}
\usepackage{standalone}
\usepackage{import}
\usepackage{xr-hyper} % Needed for external references


\usepackage{myipy2tex}
\usepackage{myrawtex}
    
\title{Hybrid Book Chapter Template}
\author{Parthiban Rajendran}

    
\begin{document}
    
    \maketitle 

    \tableofcontents

    \chapter{Introduction}
    \section{What is it about?}
    This template is to illustrate how to combine a manually written tex file and that 
    generated from a ipython notebook successfully together without conflicts. \\\\

    \textbf{Known Issues}:  \\

    In raw tex files, \\

    \begin{enumerate}
        \item If chapters might contain cross references as in this template example, always run main tex once before running any individual chapters
        \item includegraphics shall not have any width as, its restricted by default in jupyter's preamble by a hardcoded value of 80\%. If you still include, main build will fail.
        \item The float and crop options are to be specified in documentclass of main.tex just like they are in sub-files (along with more options like onesided for book type) else main build will fail. 
        \item As per standalone package rule, the sub-files should refer to main template. So do not create preamble, instead whatever you need, insert in myrawtex.sty, the common style file for raw tex.
    \end{enumerate}   
    
    In auto generated tex files from ipython notebook, \\
    
    \begin{enumerate}
        \item Remove the preamble, instead make it refer the style document created in template myipy2tex.sty 
    \end {enumerate}
   
    \chapter{Raw Tex Sample}
    \import{chapters/}{raw_sample} 
    
    \chapter{Ipython Sample}
	\import{chapters/}{ipy_sample}    

\end{document} 
